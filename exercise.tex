%#!platex main

\chapter{練習問題}

\section{正則表現・有限オートマトン}

\begin{exercise}
 次の言語を表す正則表現を示せ。
 \begin{enumerate}
  \item アルファベット$\{0, 1\}$上の記号列のうち、0から始まり1が0回以上続
	くもの全体からなる言語
  \item アルファベット$\{a, b, c\}$上の記号列のうち、1個以上の$a$と1個以
	上の$b$を含むもの全体からなる言語
  \item アルファベット$\{0, 1\}$上の記号列のうち、0と1が交互に出現するも
	の全体からなる言語
 \end{enumerate}
\end{exercise}
\begin{exercise}
 正則表現 $(0 \mid 1)^\ast 0(0\mid 1)(0\mid 1)$ が表す言語は何か。言葉で説明せよ。
\end{exercise}
\begin{exercise}
 C言語の空白記号は、空白($_{\sqcup}$)、タブ(\verb|\t|)、改行
 (\verb|\n|)が1個以上並んだ文字列である。これを正則表現で表せ。
\end{exercise}
\begin{exercise}
 C言語の16進定数は 0x あるいは 0X で始まり、数字および a, b, c, d, e, f,
 A, B, C, D, E, F が1個以上続く。これを正則表現で表せ。
\end{exercise}
\begin{exercise}
 C言語の識別子はA〜Z, a〜z, 0〜9, 下線(\underline{ })からなる文字列であ
 る。ただし、最初の文字に数字を使うことはできない。これを正則表現で表せ。
\end{exercise}
\begin{exercise}
 アルファベットを$\{0, 1\}$とするとき、次の言語を受理する有限オートマト
 ンを示せ。
 \begin{enumerate}
  \item $00$で終わる記号列全体
  \item $011$を途中に含む記号列全体
 \end{enumerate}
\end{exercise}
\begin{exercise}
 正則表現$00(0 \mid 1)^\ast$に対応する有限オートマトンを構成せよ。
\end{exercise}


\section{文脈自由文法}

\begin{exercise}
 次の文脈自由文法について、トークン列 8 - 2 + 6 の解析木を示せ。可能なす
 べての場合を示し、自然な式の意味になるものはどれか示せ。
 \[
  string \rightarrow string + string \mid string - string \mid 0 \mid 1
 \mid 2 \mid 3 \mid 4 \mid 5 \mid 6 \mid 7 \mid 8 \mid 9
 \]
\end{exercise}
\begin{exercise}
 文脈自由文法$S \rightarrow 0 S 1 \mid 0 1$はどのような言語を表すか述べよ。
\end{exercise}
\begin{exercise}
 次の文法を考える。
  \begin{eqnarray*}
   S & \rightarrow & A1B \\
   A & \rightarrow & 0A \mid \epsilon \\
   B & \rightarrow & 0B \mid 1B \mid \epsilon
  \end{eqnarray*}
 この文法に対し、文字列$00101$の最左導出、最右導出を示せ。
\end{exercise}
\begin{exercise}
 文法
 \begin{equation*}
  S \rightarrow aS \mid aSbS \mid \epsilon
 \end{equation*}
 は曖昧である。文字列$aab$に対して、解析木、最左導出、最右導出がそれぞれ
 2つずつ存在することを示せ。
\end{exercise}
\begin{exercise}
 算術式に対する文法
  \begin{eqnarray*}
   E & \rightarrow & E + T \mid T \\
   T & \rightarrow & T * F \mid F \\
   F & \rightarrow & (E) \mid {\bf id}
  \end{eqnarray*}
 を、左再帰でないように変形せよ。
\end{exercise}

\section{構文解析}

\begin{exercise}
 次の文法について以下の問いに答えよ。
 \begin{align*}
  S & \rightarrow AaAb \mid BbBa \\
  A & \rightarrow \epsilon \\
  B & \rightarrow \epsilon
 \end{align*}
 \begin{enumerate}
  \item すべての非終端記号について$\First$と$\Follow$を計算せよ。
  \item この文法はLL(1)文法か。
 \end{enumerate}
\end{exercise}
\begin{exercise}
 次の文法がLL(1)文法かどうか判定せよ。
 \begin{align*}
  S & \rightarrow aA \mid B \mid cC \mid \epsilon \\
  A & \rightarrow aB \mid Ca \\
  B & \rightarrow bC \mid d \\
  C & \rightarrow cA \mid Bc
 \end{align*}
\end{exercise}
\begin{exercise}
 次の文法がLL(1)文法かどうか判定せよ。
 \begin{align*}
  S & \rightarrow aA \mid aAb \\
  A & \rightarrow \epsilon \mid baA
 \end{align*}
\end{exercise}

\section{意味解析}

\begin{exercise}
 次の問いに答えよ。
 \begin{enumerate}
  \item 中置記法の式$9*(6+2)-4$を後置記法に変換せよ。
  \item 得られた後置記法の式をスタック1本を用いて計算することを考える。
	計算の過程を順を追って示せ。
 \end{enumerate}
\end{exercise}
\begin{exercise}
 次に示すのは、0以上の2進整数を表す文脈自由文法を基にした翻訳スキームである。
  \begin{align*}
   S & \rightarrow S' B && \{ S.val = S'.val * 2 + B.val; \}\\
   S & \rightarrow B && \{ S.val = B.val; \}\\
   B & \rightarrow 0 && \{ B.val = 0; \}\\
   B & \rightarrow 1 && \{ B.val = 1; \}
  \end{align*}
 \begin{enumerate}
  \item $11001$に対する(意味動作を含んだ)解析木を示せ。
	\label{152748_11Jul06}
  \item \ref{152748_11Jul06}で示した解析木の根の属性 $val$ を計算せよ。
  \item この翻訳スキームは何を計算するものか、説明せよ。
 \end{enumerate}
\end{exercise}

\section{実行時環境}

\begin{exercise}
 C言語変数の記憶域割付け手法を3つ示し、それぞれどのようなものか説明
 せよ。
\end{exercise}
